\documentclass[UTF8]{article}
% \usepackage{xeCJK}
\usepackage[margin=1in]{geometry}
\usepackage{cite}
\usepackage{ctex}
\usepackage{authblk}
\usepackage{siunitx}
\usepackage{amsmath}
\usepackage{graphicx}
\usepackage{subfigure}
\usepackage{hyperref}

\author{石永祥}
\affil{Peking University, Geophysics\\ Email: shiyxg@163.com}
%\mail
\date{}
\title{卷积神经网络可视化算法实现日志}
\begin{document}  
\maketitle
\section{问题简述}
卷积神经网络方法在断层识别上的应用,已经取得了较好的效果,但是之后的工作一直是在FCN等一系列方式上进行的。CNN网络的一些操作一直没有晚上,还是一年前的种种操作。在这里希望在之前的网络基础山,做一些简单的优化,查看一下识别的效果。而且做一个完整的识别可视化,包括使用反卷积(\cite{deconvnet})与另一种方法(\cite{reversion})\par
下面是对所有训练出的参数的记录日志:
\begin{itemize}
	\item[test01]test01下出来的效果不是很好,可能是因为batch——size太大或者add太小
	\item[test02]把batchSize变小,大师add没变,结果显示全部识别为了0
	\item[test03]调大了batchSize,add增加了0.1,结果就好多了
\end{itemize}

\section{DeconvNet}
有一个非常重要的点,反卷积的卷积核,到底如何获得?输入输出通道的倒置?那么另一种说法的横向倒置?或许是用于类似于卷积的方式实现反卷积吧、这一点没有弄明白,可以去参照一下原本的文章.


\section{Reversion}
看明白了,之后尝试一下。但是关键在于,他们可视化的网络只是一个识别问题,我们的网络还包含一个识别位置信息的部分,这一点可能会造成很大的问题。\par

总结一下这种可视化方式的算法:基本思想是,在网络结构确定(假定将这个关系表述为$\Gamma$)的情况下,假定输入时x,想要得知某个神经元所对应的x,可以通过梯度上升的基础上让神经元的值$\alpha_i(x)$最大,与物理问题中的反演类似,所以我称之为反演。\par
从反演的思路理解,那么基本过程可以变成:$x =x+\eta\frac{\partial\alpha_i}{\partial x}$的不断迭代的过程。但是想要生成可以使用的图片,需要有一定的优化算法:\par

\begin{itemize}
	\item decay: $x =(x+\eta\frac{\partial\alpha_i}{\partial x})\times (1-decay)$。我的理解是这种算法会逐渐削弱初始状态的影响,最终趋于一个稳定值,好处是可以快速收敛,最终只与梯度有关,而不会受初始随机状态的影响出现一堆乱码(有效信号被覆盖)。
	\item smooth: $x = Filter(x+\eta\frac{\partial\alpha_i}{\partial x})$。做平滑,是基于实际图像不会存在剧烈的震荡的假设做出
	\item clipping1: $x[where(\left|x\right|<0.001)]=0$ 将x中小值直接删除置零
	\item clipping2: $x[where(\left|x\frac{\partial\alpha_i}{\partial x}\right|<0.001)]=0$将x中对结果贡献小的部分,直接置零,方便获得$\alpha_i$的敏感区域
\end{itemize}
需要注意的是,每一个神经元对应的最大值输入时多解的,我们只是通过这种反演的方法获得这个神经元对应的一类图像,但是是无法获知\textbf{为什么}神经元对这一类图像敏感的。\par

反演出对应的图像一般有两类,一类是针对卷积神经网络的卷积层。
% \bibliographystyle{abbrv}
% \bibliography{references}
\end{document} 